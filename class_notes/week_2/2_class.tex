\documentclass[12pt]{article}

\usepackage[paper=letterpaper,margin=2.5cm]{geometry} % Set Margins

%% Math and math fonts
\usepackage{amsmath, amsthm, amssymb, amsfonts}
\usepackage{bbold} % for \mathbbm{1}

% date
\usepackage[nodayofweek]{datetime}

% Color
\usepackage{color, xcolor}

% Misc
\usepackage{environ}  % \collect@body in asmmath
\usepackage{graphicx} % \includegraphics options
\usepackage{mdframed} % text boxes
\usepackage{indentfirst} % Indent first paragraph after section header
\usepackage[shortlabels]{enumitem} % Control enumerate items with [(a)]
\usepackage{comment} % Comments
\usepackage{fancyhdr} % Headers and footers

% Tables
\usepackage{array}

% Sub-figures and figure placement
\usepackage{caption}
\usepackage{subcaption}
\usepackage{float} 

% Graphing
\usepackage{pgfplots}
\pgfplotsset{compat=1.17}
\usepackage{tikz}

% Title Placement
\usepackage{titling}
\setlength{\droptitle}{-6em}

%set indent to 
\setlength{\parindent}{0pt}
\setlength{\parskip}{10pt}

% Hyper refs
\usepackage{hyperref}
\hypersetup{
    colorlinks=true,
    linkcolor=blue,
    urlcolor  = blue,
    filecolor=magenta,      
    urlcolor=blue,
    citecolor = blue,
    anchorcolor = blue
}

% % Citation management
\usepackage{natbib}
\bibliographystyle{abbrvnat}
\setcitestyle{authordate,open={(},close={)}}

% ----------------------------------------
% TITLE
% ----------------------------------------

\pagestyle{fancy}

\lhead{Creel}
\chead{Week Two}
\rhead{FAME}

\title{FAME Week Two Class Notes -- Derivative I}
\author{Andie Creel}

\begin{document}
\maketitle

\section{Sequences and Series}

\textbf{Sequence:} The result of a function where the only inputs are integers. Any real number can be represented as a sequence. The output of a sequence is usually a set (or list) of numbers. 

\textbf{Series:} A summation of a sequence. The output of a series is typically a single number. 

\textbf{Infinity:} Arbitrarily long unit. Infinity is not a real number, it's a concept. 

\textbf{Example one: } Consider the following sequence, 
\begin{align*}
    -1, -1/2, -1/3, -1/4, -1/5,...
\end{align*}

The limit of the sequence goes to zero 
\begin{align*}
    lim \rightarrow 0
\end{align*}

\textbf{Zero:} Zero is also a unique number because (unless you're being super careful) it can lose its unit. If you say Eli has 6 apples, then you take 6 apples from him, you may not say Eli has zero \textit{apples}. You might instead say he has \textit{nothing} (which has no unit). 

\textbf{Monotone:} The function is \textit{only} increasing or \textit{only} decreasing. The function is always going in the same direction of increasing or decreasing. 

\textbf{Bound:} If the output of a function never increases above a certain number, then it is bounded from above. If it's output never decreases below a certain number, it's bounded from below. 

\textbf{Example two:} Consider the following function 
\begin{align*}
    f(x) = -1^x\\
    \text{output: -1, 1, -1, 1, -1, 1, ... } 
\end{align*}

This is an example of a sequence that is \textit{bounded} by $-1$ and $1$. However, it has \textit{ no limit} and is \textit{not monotonic.}

It's important to understand bounds, limits and monotonicity so that when you collect data you can understand the \textit{general} ways two things are related to one another. 

% \textbf{Note on limits and infinity: } You cannot \textit{really} say the limit is infinity. Limits needs to be a real number and infinity is not a real number so this is sloppy notation. 

\textbf{Method of exhaustion:} Find examples of methods of exhaustion videos and post by October 2nd. 

\textbf{Some infinities are bigger than others:} Some infinities are \textit{countable} meaning you can write down an expression that can denote any number in your sequence. However, some infinities are not countable. 

\textbf{Examples:}
Let $a$ be a real number, $0<a<\infty$,

\begin{align*}
    \lim_{N\to\infty} \frac{a}{N} = \infty\\
    \lim_{N\to\infty} \frac{N}{N+a} = 1\\
\end{align*}

Building intuition on a complicated limit: 
\begin{align*}
    \lim_{N \to \infty} \Bigg(1+ \frac{1}{N}\Bigg)^N  = e
\end{align*}

\begin{align*}
    N = 1 \implies 2 \\
    N = 2 \implies (1+1/2)(1+1/2) = 2.25\\
    N = 3 \implies (1+1/3)(1+1/3)(1+1/3) = 2\frac{4}{9}\\
    N \rightarrow \infty \implies e \approx 2.7182...
\end{align*}

Here we built intuition here by looking at the difference in output between N=1 and N=2, then N=2 and N=3. The difference between outputs became smaller as we went forward, so we intuited that this limit would be bounded above by \textit{something} (\textit{i.e.,} the limit). 

The number $e$ is \textit{irrational} and \textit{real}.

% \subsection{Why do we care? Future generations}
% If we think about sustainable development (non-declining welfare for future generations), then how many future generations do we actually care about?? \textit{All} of the future generations? Realistically, that is probably not true. You would probably prefer to keep your own children healthy more than to care about your great-great-great-grandchildren.

% If you do not prefer the immediate generation at least a little more than the future, you would let everyone starve today in order to save all resources for future generations. That doesn't seem right either. This leads us to the reason why we use discount rates, which we will discuss more later in class. 

% For example, consider a fish population that we are trying to conserve forever. If the value of that fish population is \textit{infinite}, then we would give up everything and anything else just to save the fish. That is insane. 

\section{Relating variables -- Lines}

Consider two positive real value numbers, 
\begin{align*}
    s \in S \in \mathbf{R}^{++}\\
    w \in W \in \mathbf{R}^{++}.\\
\end{align*}

If $S,W$ are two variables that you are collecting data on, then you may be measuring them with error. Say, $W$ is water, and $S$ is the survival rate of a species. 

What we're going to learn in this class is that, often times, the ``best'' way to approximate the relationship between two variables is a line. 

Notation: 
w.r.t: with respect to

\section{Intermediate Value theorem and the Definition of Derivative}
The more data you have, the better approximation you have of real relationships.

\textbf{Intermediate value theorem} tells us that we should want more data. Assuming x is continuous, we can always find another x that will help us approximate the relationship between x and y more accurately. 

Let's consider the slope. Consider the function: 
\begin{align*}
    S = f(w) 
\end{align*}
The slope is 
\begin{align*}
    \frac{f(w_2) - f(w_1)}{w_2-w_1} = \frac{\Delta S}{ \Delta w}
\end{align*}
Now consider if $w_2$ is arbitrarily smaller than $w_1$. $w_2 = w_1 + \epsilon$. We can write our slope equation as 
\begin{align*}
    \frac{f(w_1 + \epsilon) - f(w_1)}{\epsilon}
\end{align*}

This is the \textbf{definition of a derivative}, 
\begin{align*}
    \frac{ds}{dw} \equiv \lim_{\epsilon \to \ 0} \frac{f(w_1 + \epsilon) - f(w_1)}{\epsilon}
\end{align*}

\textbf{You can always use the definition of a derivative to find the derivative.} We will talk about many derivative rules, but this one will \textit{always} work.

The derivative tells us what the change in $s$ (or $f(w)$ because $s = f(w)$) is for a small change in $w$. The small change is $\epsilon$. 

Consider the function $f(x) = ax$. We can find the derivative of $f(x) = ax$ using our derivative definition. 

\begin{align}
    \frac{df(x)}{dx} &\equiv \lim_{\epsilon \to 0} \frac{f(x+ \epsilon) - f(x)}{\epsilon}\\
    &= \lim_{\epsilon \to 0} \frac{a*(x + \epsilon) - ax}{\epsilon}\\
    &= \lim_{\epsilon \to 0} \frac{ax + a\epsilon - ax}{\epsilon}\\
    &= \lim_{\epsilon \to 0} a\\
    &= a \\
    \frac{df(x)}{dx} &= a
\end{align}

So the derivative for $f(x)$ w.r.t. $x$ is a. This makes sense based on what we already know about slopes. The slope of $f(x) = ax$ is $a$. The slope tells us how a function is changing, and the derivative also tells us how the functions is changing. So the derivative of a function at a specific point is the slope of the function at that point. In high school/secondary-school, you probably heard of derivatives referred to as the tangent line.  

No matter what function you are dealing with, you can \textit{always} use the definition of a derivative to find the derivative. This rule will \textit{always} work. However, there are derivative rules that are likely worth memorizing because those rules will help you find the derivative faster than using the definition. That said, using the definition is never a wrong way of taking the derivative.  

In class, we considered the function $f(x) = -ax^2$. We used the  definition of a derivative to show that $\frac{df(x)}{dx} = -2ax$ similarly to equations 2-7.

\section{Building Intuition}
The derivative of an additive function is also additive. This is easier to see than to say. 

Consider the function 
\begin{align*}
    f(x) = g(x) + h(x)
\end{align*}
so the function $f(x)$ is two other functions of $x$ added together. The derivative of $f(x)$ is 
\begin{align*}
    \frac{df(x)}{dx} = \frac{dg(x)}{dx}  + \frac{dh(x)}{dx}
\end{align*}
so the derivative of $f(x)$ is the derivative of $g(x)$ and the derivative of $h(x)$ added together. \\

 A concrete example of this is 
\begin{align*}
    f(x) = a +bx
\end{align*}
and the derivative is 
\begin{align*}
    \frac{df(x)}{dx} &= \frac{d a }{dx} + \frac{d bx}{dx}\\
    &= 0+b\\
    &=b
\end{align*}



\section{Notation}
\subsection{First Derivative}

There are many ways to notate a first derivative
\begin{align*}
    \frac{df(x)}{dx} = f'(x) = f^1(x) = \frac{\Delta f}{\Delta x} \lim_{\epsilon \rightarrow 0} \frac{f(x+\epsilon) - f(x)}{\epsilon} = \frac{\text{rise}}{\text{run}}
\end{align*}
It is not an accident that this looks like a fraction! But derivatives are just $\frac{\text{rise}}{\text{run}}$ fractions, where the change in the denominator is really really small. 

\subsection{First Derivative w.r.t. Time}
In sustainable development, we often think about trends through time and therefore consider functions of time. We naturally are then interested in how that function \textit{changes} through time. Therefore, take the derivative of that function w.r.t time also has its own fancy notation (that comes from Newton)

\begin{align*}
    \frac{d x(t)}{dt} = \dot{x}
\end{align*}

\subsection{Second derivative}

We can also take second derivatives, which also has a few different ways to denote it:
\begin{align*}
    f''(x) = f^{(2)}(x) = \frac{d^2 f(x)}{dx^2}
\end{align*}

Example: 
\begin{align*}
    f(x) &= x^3\\
    f'(x) &= 3x^2\\
    f''(x) &= 6x\\
    f'''(x) &= 6\\
    f''''(x) &= 0
\end{align*}

\section{Power Rule}
Consider the function \[f(x) = x^N.\] The power rule says the the first derivative of $f(x)$ is 
\[\frac{d f(x)}{dx} = N x^{N-1}.\]
An example would be 
\begin{align*}
    f(x) &= x^5 \\
    \frac{df(x)}{dx} &= 5x^4
\end{align*} 

\section{Chain Rule}
The chain rule feels intimidating, but we have been using it all our lives. 

Math is about making things \textit{elegant}, not simpler. Yes, sometimes simplifying something can make it more elegant. But sometimes having an additional step is what makes the process more elegant (refer to IKEA furniture for proof, because it could benefit from having a few more steps in it). 

Consider \[F(x) = G(H(x)).\] We're interested in $\frac{dF(x)}{dx}$. We can add some steps to make finding this derivative more \textit{elegant} (although it may be less ``simple'')
\begin{align*}
    \frac{dF(x)}{dx}  = \frac{dF(x)}{dx} \times \frac{dH(x)}{d H(x)}
\end{align*}
where all we've done in this step is multiply by one, $1 = \frac{dH(x)}{d H(x)}$. We can rearrange these fractions to get the familiar chain rule. 
\begin{align*}
    \frac{dF(x)}{dx}  = \frac{dF(x)}{dG(x)}  \frac{dH(x)}{d x}
\end{align*}

while we don't easily know how to find $\frac{dF(x)}{dx}$, we do know how to find $\frac{dF(x)}{dG(x)}$ and how to find $ \frac{dH(x)}{d x}$, and the product of those two derivatives equals $\frac{dF(x)}{dx}$. 

Another way of writing this is 
\begin{align*}
    \frac{d}{dx} f(g(x)) = f'(g(x)) \cdot g'(x)
\end{align*}

\section{Product rule}
The product rule is that if $u(x)$ and $v(x)$ are differentiable functions, then the derivative of their product is given by:
\[
\frac{d}{dx} u(x)v(x) = u'(x)v(x) + u(x)v'(x)
\]

\section{Logs and derivatives}
Consider the function: 
\begin{align*}
    f(x) = \ln(x)
\end{align*}
The derivative is 
\begin{align*}
    f'(x) = \frac{1}{x}.
\end{align*}

Now consider the function 
\begin{align*}
    f(x) &= \ln(g(x)).
\end{align*}
To find the derivative we will use a log rule and the chain rule 
\begin{align}
    f'(x) &= \frac{1}{g(x)} g'(x) \\
    &= \frac{g'(x)}{g(x)} \label{perc_change}
\end{align}

This is an extremely important result! Because \ref{perc_change} is the \textbf{PERCENT CHANGE}. Percent change is a very useful statistic that we think about all the time. 

\subsection{Percent change}

\begin{align*}
    \frac{d \ln(f(x))}{dx} = \frac{f'(x)}{f(x)} = \text{percent change}
\end{align*}

Taking the derivative of the log of a function w.r.t $x$ is a great way to find the percent change. 

\textbf{But recognize how you changed the unit}: Once you have taken the log of a function, you have changed the units. You are now considering relative changes (rather than absolute changes). For many questions, knowing the relative change of you function as compared to a baseline is a very useful statistic! But recognize your question is now in terms of relative units, not absolute units. 

Relative to what (\textit{you should ask!})? Whatever is in your denominator. If $f(x)$ measures the population of wolves, you've found the percent change \textit{in the population of wolves}.

\section{Log rules and e}
Recall how we defined $e$ using a series:
\begin{align*}
    e = \lim_{N \to \infty} \left(1 + \frac{1}{N}\right)^N
\end{align*}

Now consider the function $\ln(x)$ and the definition of a derivative. We will show $\frac{d \ln(x)}{dx} = \frac{1}{x}$.

Proof
\begin{align*}
    \frac{d \ln(x)}{dx} &= \lim_{\epsilon \to 0} \frac{\ln(x+\epsilon) - \ln(x)}{\epsilon} \\
    &= \lim_{\epsilon \to 0} \frac{1}{\epsilon} \ln\left(\frac{x+\epsilon}{x}\right) \\
    &= \lim_{\epsilon \to 0} \ln\left(1+ \frac{\epsilon}{x}\right)^{\frac{1}{\epsilon}}
\end{align*}

Make up a useful substitution: $u = \frac{\epsilon}{x}$
\begin{align*}
    &= \lim_{\epsilon \to 0} \frac{1}{x} \ln\left(1+ u \right)^{\frac{1}{u}}
\end{align*}

Now make up another useful substitution: $u = \frac{1}{N}$
\begin{align*}
    &= \frac{1}{x} \ln\left(\lim_{N \to \infty}  \left(1+ \frac{1}{N}\right)^N\right)
\end{align*}

use the definition of $e$ 
\begin{align*}
    &= \frac{1}{x} \ln(e)\\
    &= \frac{1}{x}
\end{align*}

Ta da! We've used the definition of a derivative to prove our derivative rule for logs, $\ln(x)$. 

\section{Review and Examples}

\subsection{Notation:}
\begin{align}
    \frac{dF(x)}{dx} = F'(x) \\
\end{align}

\subsection{Chain rule: } Consider $F(G(X))$
\begin{align}
    \frac{dF}{dx} = \frac{dF}{dG} \frac{dG}{dx}
\end{align}

\subsection{We can work with derivatives like fractions.} Therefore, 
\begin{align}
    \frac{dF(x)}{dx} = F'(x) \implies dF(x) = F'(x) dx
\end{align}
This means we went from a \textit{derivative} to a \textit{differential}. This is an important part of notation with integrals as well. The thing to know and remember is that $dx$ just means "small changes." We can divide by small changes (like in derivatives) or multiply by small changes (like in differentials). \\

\subsection{Derivative rule for exponential function, $e^x$}:
\begin{align}
    \frac{de^x}{dx} = e^x
\end{align}
We have proportional change. This is the only function in the world where the rate of change at a point is equal to the function at that point. \\

\subsection{Example One}
Consider the function \[f(x) = e^{ax}\]

We know \[\frac{de^x}{dx} = e^x.\] 

Let's do a u-substitution to make this simpler. Let \[U(x) = ax.\]  

Our function is now written as \[f(x) = e^U.\] 

Recall the chain rule:

\begin{align}
   \frac{dF(x)}{dx} = \frac{dF(x)}{dU(x)}\frac{dU(x)}{dx} 
\end{align}

We want to know what $\frac{dF(x)}{dx}$ is. We can solve for the two simpler derivatives on the right hand side to gt this. 

\[\frac{dF(x)}{dU(x)} = e^U \]
\[\frac{dU(x)}{dx}  = a \implies\]
\[\frac{dF(x)}{dx} = a e^U \]
Now, plug our U-substitution back in: 
\[\frac{dF(x)}{dx} = a e^{ax} \]

\subsection{Example Two}
Let 
\begin{align}
    F(x) = ln(e^x)
\end{align}
We're interested in the derivative. 
\begin{align}
    \frac{dF(x)}{dx} = 1
\end{align}
Why!? Chain rule, log rule, exponential rule. First, do a u-substitution and apply the chain rule
\begin{align}
    U(x) &= e^x \implies\\
    F(x) &= ln(U(x)) \implies \\
    \frac{dF(x)}{dx} &= \frac{dF(U(x))}{dx} = \frac{dF(x)}{dU(x)}\frac{dU(x)}{dx}
\end{align}
Find $\frac{dF(x)}{dU(x)}$ using log rules 
\begin{align}
    \frac{dF(x)}{dU(x)} = \frac{1}{U(x)}
\end{align}

Find $\frac{dU(x)}{dx}$ using exponential rules 
\begin{align}
    \frac{dU(x)}{dx} = e^x
\end{align}

Now we can plug these results back into equation 11, 
\begin{align}
    \frac{dF(x)}{dx} &= \frac{1}{U(x)} e^x
\end{align}
Plug our u-substitution back in, 
\begin{align}
    &= \frac{1}{e^x} e^x \\
    &= 1
\end{align}


\end{document}