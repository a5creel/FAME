\documentclass[12pt]{article}

\usepackage[paper=letterpaper,margin=2.5cm]{geometry} % Set Margins

%% Math and math fonts
\usepackage{amsmath, amsthm, amssymb, amsfonts}
\usepackage{bbold} % for \mathbbm{1}

% date
\usepackage[nodayofweek]{datetime}

% Color
\usepackage{color, xcolor}

% Misc
\usepackage{environ}  % \collect@body in asmmath
\usepackage{graphicx} % \includegraphics options
\usepackage{mdframed} % text boxes
\usepackage{indentfirst} % Indent first paragraph after section header
\usepackage[shortlabels]{enumitem} % Control enumerate items with [(a)]
\usepackage{comment} % Comments
\usepackage{fancyhdr} % Headers and footers

% Tables
\usepackage{array}

% Sub-figures and figure placement
\usepackage{caption}
\usepackage{subcaption}
\usepackage{float} 

% Graphing
\usepackage{pgfplots}
\pgfplotsset{compat=1.17}
\usepackage{tikz}

% Title Placement
\usepackage{titling}
\setlength{\droptitle}{-6em}

%set indent to 
\setlength{\parindent}{0pt}
\setlength{\parskip}{10pt}

% Hyper refs
\usepackage{hyperref}
\hypersetup{
    colorlinks=true,
    linkcolor=blue,
    urlcolor  = blue,
    filecolor=magenta,      
    urlcolor=blue,
    citecolor = blue,
    anchorcolor = blue
}

% % Citation management
\usepackage{natbib}
\bibliographystyle{abbrvnat}
\setcitestyle{authordate,open={(},close={)}}

% ----------------------------------------
% TITLE
% ----------------------------------------

\pagestyle{fancy}

\lhead{Creel}
\chead{Week Five}
\rhead{FAME}

\title{FAME Week Five Class Notes -- Integrals I}
\author{Andie Creel}

\begin{document}
\maketitle

\section{A very simple integral}

Think about if you have a function that is just equal to a constant $f(x) = A$. You want to find the area under your curve (\textit{i.e.,} take an integral), where the domain is $x=0$ through $x =10$ (which can also be written as $x \in [0, 10]$). The integral of $f(x)$ is the area under the constant $A$, 

\begin{align*}
    \int_{0}^{10} f(x) dx &= \int_{0}^{10} A dx
\end{align*}

Constants can be moved in front of the integral signs, the right-hand side above can be rewritten  
\begin{align*}
    &= A\int_{0}^{10} dx\\
    &= A * (x |_0^{10})\\
    &= A * (10 - 0) \\
    &= A * 10. 
\end{align*}

When a function is equal to a constant, it's just a line. The area under a line is a rectangle. The area of a rectangle is width times height. The width in this example is the length of the domain, 10, and the height is the constant $A$. 

\section{Reimann Sums}
An integral is the area under a curve. A reimann sum approximates this area by using rectangles of different heights whose width gets smaller and smaller (\textit{i.e.,} width goes to zero).

Let $A(\cdot)$ be an area function. $A$ returns the area under a curve. 

Recall that small changes in $x$ are equal to $\epsilon$ but also can be written as $\partial x$, $\Delta x = \epsilon = \partial x$.

\begin{align*}
    f(a) = \frac{A(a+\epsilon)-A(a)}{\epsilon}
\end{align*}

If we take the limit of this function as $\epsilon \to 0$, we get the definition of a derivative 
\begin{align*}
    \lim_{\epsilon \to 0} f(a) = \frac{A(a+\epsilon)-A(a)}{\epsilon} \implies\\
    A'(x) = \frac{\partial A}{\partial x} \implies \\
    A'(x) \partial x = \partial A \implies \\
    A'(x) \epsilon = \partial A 
\end{align*}
where all of these steps come from knowing $\epsilon = \partial x$ and treating a derivative like a fraction. What is $A'(x)$? If $A(\cdot)$ is the area under the curve, then the derivative is the curve which is $f(x)$ therefore $A'(x) = f(x)$. 

Therefore, a change in area can be written as
\begin{align*}
    \partial A = f(x) \epsilon
\end{align*}
which is just a rectangle where $f(x)$ is the height and $\epsilon$ is the width. 

We can rewrite this as 
\begin{align*}
    \partial A =  \sum_{x=a}^{b}f(x) \epsilon.
\end{align*}
if you want to find the area under the curve from the point where $x =a $ to the point where $x = b$. This is a sum of rectangles. 

Integration is just fancy summation! So, as $\epsilon$ gets really small, and we move from discrete changes to arbitrarily small changes (smooth). We can then write this as 
\begin{align*}
    \partial A =  \int_{a}^{b} f(x) \partial x.
\end{align*}

\section{Additively separable}
Consider a function 
\begin{align*}
    f(x) = h(x) + g(x)
\end{align*}
where $f(x)$ can be separated into two simpler functions. For instance, any polynomial would take this form (\textit{e.g.} $f(x) = 3x^2 + 6x$). 

The function $A(\cdot)$ is still returning the area under the curve $f(x)$. Then, we can write the function as 

\begin{align*}
    dA = \int \Big(h(x) + g(x) \Big)dx
\end{align*}

Because the integral is a linear operator, we do not need to worry about Jensen's inequality and we can pass the integral sign through the addition 
\begin{align*}
    dA = \int h(x) dx + \int g(x) dx
\end{align*}

\section{Considering constants}
Remember that when you take the derivative of a constant, it's zero. When you integrate a function, you will not be able to get the constant back out (you don't know what the constant is),

\begin{align*}
    F(x) + C = \int f(x) dx.
\end{align*}
When ever you take an integral, remember to add on your unidentified constant $C$. \\

If you're doing a \textbf{definite} integral meaning you have the bounds $a$ and $b$ you do NOT need to worry about the unidentified constant $C$ because it will difference out. However, if you are taking an \textbf{indefinite} integral then we don't know the bounds, and so we do need to keep track of the unidentified constant.

\section{Example of constants}

Consider this the function $f(x) = 3$, 
\begin{align*}
    F(x) &= \int f(x) dx\\
    &=  \int 3 dx\\
    &= 3 \int dx \\
    &= 3 \int 1 dx\\
    &= 3x + C
\end{align*}

% \section{Derivative rules to remember}

% \begin{align*}
%     f(x) = X^n \implies f'(x) = Nx^{N - 1}\\
%     f(x) = e^x \implies f'(x) = e^x
% \end{align*}

\section{Considering ``Additively Separable'' functions and constants}

In this example, we will see how the constants from separate terms of the integral can be combined into one final constant. 

Consider the function 
\begin{align*}
    f(x) = \int 7x^2 + 3x +6 dx.
\end{align*}

The integral is a "linear operator" so we can integrate each of these terms that are separated by addition signs separately, 

\begin{align*}
    f(x) = \int 7x^2 dx + \int 3x dx +\int 6dx.
\end{align*}
We can now use the integral's version of the power rule to integrate these. 

\textbf{Integral power rule:}

\begin{align*}
    \int x^n dx = \frac{1}{n}x^{n+1} + C
\end{align*}

applying the power rule to each term gives us 
\begin{align}
    f(x) = (\frac{7}{3} x^3 + X) + (\frac{3}{2} x^2 + Y) + (6x + Z) \label{const}
\end{align}
where $X, Y, Z$ are our unknown constants. We can define another unknown constant as 
\begin{align*}
    C = X + Y + Z 
\end{align*}
and rewrite equation \eqref{const} as 
\begin{align*}
    f(x) = \frac{7}{3} x^3  + \frac{3}{2} x^2 + 6x + C. 
\end{align*}

This shows us that even though each separate term we're integrating may have its own constant, we can combine them into a single final constant during integration. 

\section{Integration by U-substitution}

U-substitution is the reverse of the chain rule for derivatives. The key idea: when you see a function and its derivative multiplied together, you can do a U-substitution to simplify the integral.

\textbf{The general rule:}
\begin{align*}
    \int f(u) \frac{du}{dx}dx = \int f(u) du
\end{align*}

\textbf{Step-by-step strategy:}
\begin{enumerate}
    \item \textbf{Identify the inner function:} Look for a ``complicated'' part of your integrand that, when differentiated, appears elsewhere in the integrand
    \item \textbf{Set $u$ equal to that inner function:} $u = \text{[inner function]}$
    \item \textbf{Find $du$:} Take the derivative: $du = \text{[derivative of inner function]} \cdot dx$
    \item \textbf{Solve for $dx$:} Rearrange to get $dx = \frac{du}{\text{[derivative of inner function]}}$
    \item \textbf{Substitute everything:} Replace all $x$ terms with $u$ terms
    \item \textbf{Integrate:} Solve the simpler integral in terms of $u$
    \item \textbf{Substitute back:} Replace $u$ with the original expression
\end{enumerate}

\textbf{Example}
\begin{align*}
    f(x) = \int x e^{x^2} dx
\end{align*}

\textbf{Step 1 \& 2:} The inner function is $x^2$ (it appears in the exponent), so let $u = x^2$

\textbf{Step 3:} Find $du$: $du = 2x dx$

\textbf{Step 4:} Solve for $dx$: $dx = \frac{du}{2x}$

\textbf{Step 5 \& 6:} Substitute and integrate:
\begin{align*}
   &= \int x e^{x^2} dx \\
    &= \int x e^{u} \frac{du}{2x} \quad \text{(substitute $u = x^2$ and $dx = \frac{du}{2x}$)}\\
    &= \int \frac{e^u}{2} du \quad \text{(the $x$ terms cancel)}\\
    &= \frac{1}{2} \int e^u du \text{ (constant moves out front)} \\
    &= \frac{1}{2} e^u + C
\end{align*}

\textbf{Step 7:} Substitute back: $u = x^2$, so our final answer is:
\begin{align*}
    f(x) = \frac{1}{2} e^{x^2} + C
\end{align*}

\textbf{Key insight:} Notice how the $x$ from outside the exponential was exactly what we needed to cancel with the $2x$ from $du = 2x dx$. This is the hallmark of a good u-substitution - the pieces should cancel in useful ways to simplify the integration.

\section[The importance of discounting: e to the power of -rt]{The importance of discounting: $e^{-rt}$}


Consider a funny integral: 
\begin{align*}
    \int_0^\infty e^{-rt}dt
\end{align*}
Why is this weird? Because the support is 0 to $\infty$. However, when we integrate it, we will see that the area under this curve is finite (despite the function going to infinity). Let's show this is true with a U-sub. \\

Let 
\begin{align}
    U = -rt \label{U}\\
    \frac{dU}{dt} = -r \implies \\
    dU = -r dt \implies \\
    \frac{dU}{-r} = dt \label{dt}
\end{align}

plug in \ref{U} and \ref{dt} into our original integral 
\begin{align*}
    &= \int_0^\infty - \frac{1}{r} e^U du\\
    &= -\frac{1}{r} \int_0^\infty e^u du\\
    &= -\frac{1}{r} e^u \bigg|_0^\infty \\
    &= -\frac{1}{r} e^{-rt} \bigg|_0^\infty \\
    &=  -\frac{1}{r} e^{-r\infty} - -\frac{1}{r} e^{-r 0} \\
    &= \frac{1}{r} ( -e ^{-r\infty} + e^{-r0} \\
    &= \frac{1}{r} (0 + 1) \\
    &= \frac{1}{r}
\end{align*}

Even though we integrated over an infinite support (0 to $\infty$) we have a finite area. \\

This is \textbf{exponential decay} or a \textbf{discounting process}. This is a very important integral and it's used in applied settings often. \\

This discount factor means that we can project a program into the future \textit{infinitely} and still find a finite value for the impact of that program. We can compare a program from now to the end of the world to a another program for the same length of time and see how they differ. \\

For example:
\begin{align*}
    X = \int_0^\infty M(t) e^{-rt} dt\\
    Y = \int_0^\infty N(t) e^{-rt} dt
\end{align*}

We can compare $X$ and $Y$ over an arbitrarily long period. If $M(t)$ represented a policy that affected wellbeing, and $N(t)$ was an alternative policy, we could use the discount rate $e^{-rt}$ to be able to evaluate which of those programs would achieve sustainable development goals over an arbitrarily long period.\\

\subsection{Thin and Fat Tails}


\begin{figure}[htp]
    \centering
        \includegraphics[width=0.5\textwidth]{Screen Shot 2023-10-04 at 10.56.06 AM.png}
    \caption{https://www.statisticshowto.com/fat-tail-distribution/}
    % \label{fig:sample}
\end{figure}

Above, if you were to integrate the "heavy tail" function it would not be finite! As x goes to $\infty$ the area under the function keeps getting bigger and bigger. If you were to integrate the "heavy tailed" function it would "blow up" aka go to infinity. \\

The if you were to integrate under the "exponential" function in the picture above, you would get a finite answer similar to the function $e^{-rt}$. \\







\end{document}