\documentclass[12pt]{article}

\usepackage[paper=letterpaper,margin=2.5cm]{geometry} % Set Margins

%% Math and math fonts
\usepackage{amsmath, amsthm, amssymb, amsfonts}
\usepackage{bbold} % for \mathbbm{1}

% date
\usepackage[nodayofweek]{datetime}

% Color
\usepackage{color, xcolor}

% Misc
\usepackage{environ}  % \collect@body in asmmath
\usepackage{graphicx} % \includegraphics options
\usepackage{mdframed} % text boxes
\usepackage{indentfirst} % Indent first paragraph after section header
\usepackage[shortlabels]{enumitem} % Control enumerate items with [(a)]
\usepackage{comment} % Comments
\usepackage{fancyhdr} % Headers and footers

% Tables
\usepackage{array}

% Sub-figures and figure placement
\usepackage{caption}
\usepackage{subcaption}
\usepackage{float} 

% Graphing
\usepackage{pgfplots}
\pgfplotsset{compat=1.17}
\usepackage{tikz}

% Title Placement
\usepackage{titling}
\setlength{\droptitle}{-6em}

%set indent to 
\setlength{\parindent}{0pt}
\setlength{\parskip}{10pt}

% Hyper refs
\usepackage{hyperref}
\hypersetup{
    colorlinks=true,
    linkcolor=blue,
    urlcolor  = blue,
    filecolor=magenta,      
    urlcolor=blue,
    citecolor = blue,
    anchorcolor = blue
}

% % Citation management
\usepackage{natbib}
\bibliographystyle{abbrvnat}
\setcitestyle{authordate,open={(},close={)}}

% ----------------------------------------
% TITLE
% ----------------------------------------

\pagestyle{fancy}

\lhead{Creel}
\chead{Week Five}
\rhead{FAME}

\title{FAME Week Six Class Notes -- Integrals II}
\author{Andie Creel}

\begin{document}
\maketitle

\section[The importance of discounting: e to the power of -rt]{The importance of discounting: $e^{-rt}$}


Consider a funny integral: 
\begin{align*}
    \int_0^\infty e^{-rt}dt
\end{align*}
Why is this weird? Because the support is 0 to $\infty$. However, when we integrate it, we will see that the area under this curve is finite (despite the function going to infinity). Let's show this is true with a U-sub. 

Let 
\begin{align}
    U = -rt \label{U}\\
    \frac{dU}{dt} = -r \implies \\
    dU = -r dt \implies \\
    \frac{dU}{-r} = dt \label{dt}
\end{align}

plug in \ref{U} and \ref{dt} into our original integral 
\begin{align*}
    &= \int_0^\infty - \frac{1}{r} e^U du\\
    &= -\frac{1}{r} \int_0^\infty e^u du\\
    &= -\frac{1}{r} e^u \bigg|_0^\infty \\
    &= -\frac{1}{r} e^{-rt} \bigg|_0^\infty \\
    &=  -\frac{1}{r} e^{-r\infty} - -\frac{1}{r} e^{-r 0} \\
    &= \frac{1}{r} ( -e ^{-r\infty} + e^{-r0} \\
    &= \frac{1}{r} (0 + 1) \\
    &= \frac{1}{r}
\end{align*}

Even though we integrated over an infinite support (0 to $\infty$) we have a finite area. 

This is \textbf{exponential decay} or a \textbf{discounting process}. This is a very important integral and it's used in applied settings often. 

This discount factor means that we can project a program into the future \textit{infinitely} and still find a finite value for the impact of that program. We can compare a program from now to the end of the world to a another program for the same length of time and see how they differ. 

For example:
\begin{align*}
    X = \int_0^\infty M(t) e^{-rt} dt\\
    Y = \int_0^\infty N(t) e^{-rt} dt
\end{align*}

We can compare $X$ and $Y$ over an arbitrarily long period. If $M(t)$ represented a policy that affected wellbeing, and $N(t)$ was an alternative policy, we could use the discount rate $e^{-rt}$ to be able to evaluate which of those programs would achieve sustainable development goals over an arbitrarily long period.

\subsection{Thin and Fat Tails}


\begin{figure}[htp]
    \centering
        \includegraphics[width=0.5\textwidth]{Screen Shot 2023-10-04 at 10.56.06 AM.png}
    \caption{https://www.statisticshowto.com/fat-tail-distribution/}
    % \label{fig:sample}
\end{figure}

Above, if you were to integrate the "heavy tail" function it would not be finite! As x goes to $\infty$ the area under the function keeps getting bigger and bigger. If you were to integrate the "heavy tailed" function it would "blow up" aka go to infinity. 

The if you were to integrate under the "exponential" function in the picture above, you would get a finite answer similar to the function $e^{-rt}$. 

\section{Integration by Parts}
It's not as bad as you think it is! Previously, integration by substitution is similar to the chain rule for derivatives. Integration by parts is \textit{kind of} like the product rule for derivatives. 

Integration by parts is not really an integral trick, it's more like an algebra trick. The following example is how to derive the integration by parts rule, which mostly is a bunch of algebra tricks.

\subsection{Deriving Integration by Parts}

Consider the function 
\begin{align*}
    z(x) = f(x) g(x)
\end{align*}

We want the integral of $z(x)$ 
\begin{align*}
    Z(x) = \int z(x) = \int f(x)g(x) dx. 
\end{align*}
Notice that I am defining capital $Z(x)$ as the integral of the lower case $z(x)$. This is a pretty common notation. 


Now, we know how to take the derivative of $z(x)$ with respect to $x$, $dz(x) /dx$, using the product rule. We'll then integrate both sides
\begin{align*}
    \frac{dz(x)}{dx} = f'(x) g(x) + f(x) g'(x) \implies\\
    \int dz = \int (f'(x) g(x) + f(x) g'(x)) dx 
\end{align*}

Remember integrals are linear operators, and distribute the integral through the right-hand side:
\begin{align*}
    \int dz = \int f'(x) g(x) dx + \int f(x) g'(x) dx\\
\end{align*}

Note that $\int dz = z(x) = f(x) g(x)$ because of how we originally defined our $z(x)$ function. So substitute that equivalence on the left-hand side, then do some algebra:
\begin{align}
    f(x) g(x) = \int f'(x) g(x) dx + \int f(x) g'(x) dx \implies \\
    f(x) g(x) - \int f(x) g'(x) dx = \int f'(x) g(x) dx \\
    f(x) g(x) - \int f(x) g'(x) dx = \int g(x) f'(x) dx  \\
    \int g(x) f'(x) dx = f(x) g(x) - \int f(x) g'(x) dx \label{eqn_1}
\end{align}

Next, define a $U$ and a $V$ and their respective derivatives: 
\begin{align*}
    U = g(x) \\
    dV = f'(x)dx \\
    V = f(x) \\
    dU = g'(x)dx
\end{align*}

Plug these back into Equation \ref{eqn_1} 
\begin{align}
    \int U dV = V U - \int V dU \label{eqn:int_parts}.
\end{align}
Equation \ref{eqn:int_parts} is the equation for integration by parts. 

\subsection{Tips}
How can you remember the order of what should be U and what should be dV? 

U $\rightarrow$ Log $\rightarrow$ Inverse trig $\rightarrow$ Algebraic $\rightarrow$ Trig $\rightarrow$ Exponential $\rightarrow$ dV 


\subsection{Example}
Consider the function 
\begin{align}
    \int x  e^x dx
\end{align}

Whats's U and what's V? 
\begin{align}
    \int U dV = vu - \int V dU\\
    U = x\\
    dU = dx \\
    V = e^x\\
    dV = e^x dx
\end{align}

Rewrite the whole thing! And use integration by parts
\begin{align}
    \int x e^x dx &= e^x x - \int e^x dx \\
    &= x e^x - e^x \\
    &= (x - 1) e^x
\end{align}

\section{Probability density function}

Consider the random variable $Y$. Random variables aren't random and they aren't variables. They're really more like functions. 

\subsection{Normal Distribution}
If $Y$ is distributed normally it will look like 
\begin{align}
    Y = f(y) = \int_{- \infty}^{\infty} \frac{1}{\sigma \sqrt{2 \pi}} e^{-\frac{1}{2}(\frac{y - \mu}{\sigma})^2} dy = 1
\end{align}
the inside of the integral is the probability density (pdf) function. The integral of a pdf must always equal 1. 

\subsection{Another pdf example}
Consider the random variable 
\begin{align}
    Y = f(y) = 2y e^{-y^2}
\end{align}
and the support is $[0, \infty )$. Our probability density function (pdf) is $f(y)$. 

Our cumulative distribution function (CDF) is the integral of our pdf 
\begin{align}
    CDF &= \int_0^\infty 2y e^{-y^2} dy \\
    &= 2 \int_0^\infty y e^{-y^2} dy
\end{align}

Let $u = -y^2$, $du = -2y dy$
\begin{align}
    &= 2 \int_0^\infty  ye^u \frac{du}{-2y} \\
    &= -1  \int_0^\infty e^u du \\
    &= -1 e^u \bigg |_0^\infty \\ \label{eqn:one}
    & = -1 e^{-y^2}\bigg |_0^\infty \\ 
    & = 0 - - 1\\
    & = 1
\end{align}

Which integrates to 1 after doing a u substitution, and so we know we have a proper pdf. 

\subsection{Finding median with PDF}

What if you want to find the \textbf{median}? Set the CDF equal to $0.5$ because the median is the 50th percentile. 

The unknown is the point on the x-axis where the area under the pdf is equal to 0.5 (which is defined as our median). Start with equation \ref{eqn:one} from our previous integral derivation because we know the integral is equal to that. But replace the upper bound to an unknown $\tilde{y}$ and set equal to 0.5.

\begin{align}
    -1 e^{-y^2} \bigg |_0^{\tilde {y}} &= 0.5\\
    -e^{- \tilde y ^2} - - 1 &= 0. 5\\
    -e^{- \tilde y ^2} &= - 0.5 \\  
    e^{- \tilde y ^2} & = 0.5\\
    ln( e^{- \tilde y ^2}) &= ln(1/2) \\
    - \tilde y^2 &= ln(1/2)  \\
    \tilde y^2 &= - ln(1/2) \\
    \tilde y &= \sqrt{-ln(1/2)}
\end{align}
so we know that the median value is $\sqrt{- ln(1/2)}$.


\subsection{Finding the mean with the PDF}
To find a mean, we sum the values and divide it by the number of people we summed over. This is the same as multiplying everyone's value by $\frac{1}{N}$. 
\begin{align}
   E[Y]= \bar y &= \frac{1}{N}\sum_n y_n\\
   &= \sum_n \frac{1}{N} y_n
\end{align}


We can use the pdf to get our mean! The pdf would replace the $\frac{1}{N}$ and the sum would be replaced by the integral. 

\begin{align}
    E[Y]= \bar y = \int_0^\infty 2ye^{-y^2} * y dy
\end{align}

Important: I let the pdf be $f(y)$ for the mean and variance examples.
\begin{align}
    f(y) = pdf(y)
\end{align}

The general rule for finding the mean of a random variable by using the pdf $f(y)$ of that random variable is 
\begin{align}
    E[Y] = \int_\Omega y f(y) dy
\end{align}
where $\Omega$ is the support of $y$. 


\section{Integrating logs}

Consider the integral of $1/x$, which we know from knowing the derivative of $ln(x)$,
\begin{align}
    \int \frac{dx}{x} = \int \frac{1}{x} dx = ln(x)
\end{align}
Cool! Wait, but what the integral of a log,
\begin{align}
    \int ln(x) dx = ??
\end{align}
This was a really hard problem that existed for a very long time in mathematics. What's the trick? To use integration by parts, we consider this function instead 

\begin{align}
    \int ln(x) 1 dx 
\end{align}
Now we can use integration by parts and make the following variable assignments,

\begin{align}
    U = ln(x) \\
    dU = \frac{1}{x} dx\\
    V = x \\
    dV = 1 dx\\
\end{align}
\textbf{Do NOT forget the $dx$ when you get $dU$!!} Now, plug it all back into our integration by parts formal 
\begin{align}
    \int U dV &= V U - \int V dU\\
    \int ln(x) 1 dx &= x ln(x) - \int x \frac{1}{x} dx \implies \\
    &= x ln(x) - \int 1 dx  \\
    & = x ln(x) - x\\
    &= x (ln(x) - 1)
\end{align}
You can take the derivative of this final equation to check and you'll find the derivative of that final equation does equal $ln(x)$.

\section{Cross Partials Derivatives to Double Integrals}

We're going to take a \textbf{cross partial} derivative of a function (meaning derivative with respect to one variable and then derivative wrt a different variable). 

We then are going to do a \textbf{double integral} to reverse these two derivatives. 

Consider the function 
\begin{align}
    F(x,y) = x^\alpha y^\beta \label{org}
\end{align}

Take the derivative with respect to $x$,
\begin{align}
    \frac{\partial F}{\partial x} = \alpha x^{\alpha -1}y ^\beta \label{derv}
\end{align}

Now, take the derivative of $\partial F / \partial x$ with respect to $y$, 
\begin{align}
    \frac{\partial F}{\partial x \partial y} = F_{x,y}(x,y) = \alpha \beta x^{\alpha -1} y^{\beta - 1}
\end{align}

Now, we can do a \textbf{double integral} to undo the two derivatives we've taken (cross partial). 

\begin{align}
    \int_x \int_y \alpha \beta x^{\alpha -1} y^{\beta - 1} dy dx
\end{align}
First, integrate with respect to y. To do so, we can pull the constants out front of the integral that has to do with variable $y$,
\begin{align}
   = \int_x \alpha \beta x^{\alpha -1}\int_y  y^{\beta - 1} dy dx\\
   = \int_x \alpha \frac{\beta}{\beta} x^{\alpha -1} y^\beta dx\\
   = \int_x \alpha x^{\alpha -1} y^\beta dx \label{int}
\end{align}

We have undone the derivative wrt to $y$ by integrating wrt $y$! Note that the term in the integral in equation \ref{int} is the same as \ref{derv}. We can now undo the integral wrt to $x$ by doing the integral wrt to $x$. 

\begin{align}
    = \frac{\alpha}{\alpha} x^\alpha y^\beta + C\\
    = x^\alpha y^\beta + C
\end{align}
and we have successfully returned to $F(x,y) = x^\alpha y^\beta +C$ which is the same as our original equation \ref{org} (plus an arbitrary constant).

\section{Double Integral}
We need to sum over one variable and then we sum over the other.
\begin{align}
    \int \int xy dx dy = \int \bigg(\int xy dx\bigg) dy 
\end{align}
We can solve the integral in parentheses first while treating $y$ like a constant 
\begin{align}
    &= \int \bigg( y\int x dx\bigg) dy \\
    &= \int \bigg( y\frac{1}{2} x^2 \bigg) dy
\end{align}
Now we can treat $x$ as a constant
\begin{align}
    &= \frac{1}{2} x^2 \int y dy\\
    & = \frac{1}{2} x^2 * \frac{1}{2} y^2 + C\\
    &= \frac{1}{4} x^2 y^2 + C
\end{align}



\end{document}